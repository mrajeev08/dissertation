\pagenumbering{gobble}

\begin{center}

\copyright  Copyright by Malavika Rajeev, 2021. All rights reserved.

\end{center}

\frontmatter

\setcounter{page}{3}  % start page numbering at three for Princeton

\section*{Abstract}

Canine rabies is responsible for an estimated 60,000 human deaths annually across the globe. These deaths are entirely preventable, either through mass dog vaccination or by post-exposure vaccination of humans, and the World Health Organization and its partners have set a goal of zero human deaths due to canine rabies by the year 2030. My dissertation uses two different methodologies, field epidemiological studies and mathematical modeling, to answer questions of how we can improve control and surveillance for canine rabies. Chapters 1 and 2 of my dissertation focus on epidemiological studies of rabies in Madagascar. In Chapter 1, I describe the results of my project in collaboration with in-country partners (the Ministry of Health, the Department of Veterinary Services, and the Institut Pasteur de Madagascar) to collect baseline data on rabies incidence in animals and rabies exposures and deaths in humans, and to test better methods for surveillance in the Moramanga District, Madagascar. In Chapter 2, I use this dataset and other data on patients seeking care for animal bites in Madagascar to estimate how geographic accessibility to the human rabies vaccine drives the burden of rabies spatially, and to explore how improving accessibility could mitigate this burden. In the second half of my dissertation, I look more broadly at how transmission modelling of canine rabies can be applied to answer critical questions regarding rabies control. In Chapter 3, I critically review the existing modeling literature, identify gaps in the current methods, and propose new ways forward for how modeling can contribute to rabies control. In Chapter 4, I use fine scale spatial and temporal data on rabies cases in the Serengeti District, Tanzania to test dynamic models of rabies transmission and identify key features of transmission and control necessary to recapture dynamics. Overall, this work brings forth a set of epidemiological and quantitative toolsets that can be used to tackle key challenges on the road to global rabies elimination. 

\newpage

\section*{Acknowledgements}

There are too many people to name here that made this dissertaton happen, but I will try. First, thank you to my advisor Jess Metcalf and my unofficial co-advisor Katie Hampson for guiding me with such kindness and good humour, for your patience, and for being the type of advisors that you can cry about bed bugs to from an internet cafe in Madagascar. Thank you also to my committee, Bridgett vonHoldt and Bryan Grenfell for their feedback, wisdom, and support throughout this process. 

The work in Madagascar would not have been possible without the expertise, time, and generosity of numerous local clinicians, veterinarians, and livestock officers in the Moramanga District and my colleagues and collaborators at the Ministry of Public Health, Institut Pasteur de Madagascar, The Mad Dog Initiative, and Traveling Animal Doctors. In particular, I would like to thank Glenn Edosoa, Chantal Hanitriniana, Soa Fy Andriamandimby, Jean Hyacinthe Randrianarisoa, Ranaivoarimanana, Fierenantsoa Randriamahatana, Esther Noiarisaona, Nely Jose, Adrimananjara Mamitiana, Jean Michel Heraud, Helene Guis, Rila Ratovoson, Cara Brook, Christian Ranaivoson, John Friar, Kim Valenta, Zachary Farris, Angelique Ferreira, Zoavina Randriana, Radoniaina Rafaliarison, Jochem Ladstrager, and Tsiky Rajaonarivelo.

I also owe my start in dog rabies to Katie and her wonderful team and inspiring projects in Tanzania and Glasgow. I'm particularly grateful to Ahmed Lugelo, Zilpah Kaare, Renatus, Rafael, Mzee Magoto, Kennedy Lushasi, Maganga Sambo, and Anna Czuprynna. I also greatly appreciate the friendly faces and pub visits during my short stays in Glasgow at Boyd Orr, and in particular thanks to Christina Faust and Colin Torney (and Liam), Jo Halliday, Daniel Streicker, Liliana Salvador, Isty Rysava, and Laurie Baker for hosting/and or entertaining me at various times. I am also grateful to Charlie and Jack Champson (and Nai)-- without whom I would not have landed my first rabies gig! 

To my current and former Metcalf lab mates, I am so grateful for your comraderie, expertise, and sanity checks. Thanks especially to my office mates and travel buddies: Ayesha Mahmud, Amy Winter, Saki Takahashi, Joaquin Prada, Keitly Mensah, Amy Wesolowski, Brooke Bozick, Benny Rice, Fidy Rasambainarivo, Ian Miller, Ryo Kinoshita, Marjolein Bruijning, Sam Huberman, Bertie, and Svevo. Thanks also to Ros and Peter Metcalf and Chez Metcalf for hosting me during my first trips to Madagascar.

The Department of Ecology and Evolutionary Biology, in general, has been a great home to build my dissertation. From Disease Group meetings to beer hour, I have benefited from so many informal and formal conversations with some superb scientists and people. In terms of getting the nuts and bolts of science done, I have to thank Sandy Kominski, Sandra Kaiser, Lolly O'Brien, Diane Carlino, Jonathan Polk, Ksenia Rodionova, Jesse Saunders, and all the other great folks that make the department hum. In terms of getting the fun done, I am so grateful to my fellow graduate students and post-docs who make/made up EEB (and in particular thanks to Wen, Luojun, Kaia, Mayank, Ruthie, Char, Tilman, Caroline, Nando, and many many more). Thanks also to all my fellow Community Associates and to Dean Lily Secora for pulling me out from my departmental bubble on a regular basis. A big thanks to my friends in Madagascar (in particular my various fantastic roomates Bija, Mel, Kitty, Caroline, Dorothee) and to my friends outside of Princeton (in particular Kelly + Dugan, Sara DLT, Serena, Molly, Ottman) for their check-ins and moral support.

It perhaps isn't too strange for this dissertation to honor a dog or two: Tulip, Burlap, Ramu, Sumi, Jukebox, Piper, Sally, Kapiki, (+ Wulfie and Hawkeye, two honorary cat-dogs), thank you so much for your company past and present. And of course to Dolly, current and cutest canine love of my life. 

None of this would have been possible without my family. I am grateful to my extended family for their love and support, and for welcoming me home whenever I get the chance to come back. Many thanks to the Salkowski's, as well, for welcoming me into their's. Thanks to Seth for the indirect phone conversations and hobbyist inspirations. I don't know where I would be in life, let alone this dissertation, without my parents, Sree Rajeev and Rajeev Nair, and my sister Meenakshi Rajeev. And of course, to David Salkowski, I could not have possibly gotten here in one piece and with such a full heart, without you. 

This work was funded through the National Science Foundation GRFP, the Center for Health and Wellbeing at Princeton University, the Department of Ecology and Evolutionary Biology, and the Princeton Insitute for International and Regional Studies. Much of this work would not have been possible without the Research Computing resources available through the Princeton Institute for Computational Science and Engineering (PICSciE). In addition, I am eternally grateful to the Rstats community online and to Stack Exchange for being my programming instructors and constant companions, and broadly to the open-source software developers without whom this work would have likely taken me 10 - 100 lifetimes.

Finally, I am indebted to all the folks out in the world vaccinating dogs and people, collecting samples, doing laboratory diagnostics--doing the hard work daily to try and make the end of rabies a reality. 
