\pagenumbering{gobble}

\begin{center}

\copyright  Copyright by Malavika Rajeev, 2021. All rights reserved.

\end{center}

\frontmatter

\setcounter{page}{3}  % start page numbering at three for Princeton

\section*{Abstract}

Canine rabies is responsible for an estimated 60,000 human deaths annually across the globe. These deaths are entirely preventable, either through mass dog vaccination or by post-exposure vaccination of humans, and the World Health Organization and its partners have set a goal of zero human deaths due to canine rabies by the year 2030. My dissertation uses two different methodologies, field epidemiological studies and mathematical modeling, to answer questions of how we can improve control and surveillance for canine rabies. Chapters 1 and 2 of my dissertation focus on epidemiological studies of rabies in Madagascar. In Chapter 1, I describe the results of my project in collaboration with in-country partners (the Ministry of Health, the Department of Veterinary Services, and the Institut Pasteur de Madagascar) to collect baseline data on rabies incidence in animals and rabies exposures and deaths in humans, and to test better methods for surveillance in the Moramanga District, Madagascar. In Chapter 2, I use this dataset and other data on patients seeking care for animal bites in Madagascar to estimate how geographic accessibility to the human rabies vaccine drives the burden of rabies spatially, and to explore how improving accessibility could mitigate this burden. In the second half of my dissertation, I look more broadly at how transmission modelling of canine rabies can be applied to answer critical questions regarding rabies control. In Chapter 3, I critically review the existing modeling literature, identify gaps in the current methods, and propose new ways forward for how modeling can contribute to rabies control. In Chapter 4, I use fine scale spatial and temporal data on rabies cases in the Serengeti District, Tanzania to test dynamic models of rabies transmission and identify key features of transmission and control necessary to recapture dynamics. Overall, this work brings forth a set of epidemiological and quantitative toolsets that can be used to tackle key challenges on the road to global rabies elimination. 
